\documentclass[12pt,a4paper,twoside,openany]{report}

\usepackage[paper=a4paper, left=2cm, right=2cm, top=2cm, bottom=2cm]{geometry}

\usepackage{helvet}

\usepackage[usenames,dvipsnames]{color}

\usepackage{hyperref}

\usepackage{remreset}
\makeatletter
\@removefromreset{footnote}{chapter}
\makeatother

\usepackage{setspace}
\onehalfspace

\usepackage[table]{xcolor}

\usepackage{listings}
\lstloadlanguages{PHP}
\lstset{	language=PHP ,
			alsolanguage=SQL ,
			morecomment=[s]{/*}{*/},
			morecomment=[s]{/**}{*/} ,
			morestring=[b]' ,
			morestring=[b]{''} , 
			keywords=[1]{} ,
			breaklines=true ,
			breakatwhitespace=true ,
			showstringspaces=false ,
			basicstyle=\ttfamily\small ,
			commentstyle=\color{Gray} ,
			numbers=left ,
			numberstyle=\tiny ,
			stepnumber=1 ,
			numbersep=5pt ,
			tabsize=4}

\usepackage{footmisc}
\renewcommand\footnotelayout{\fontfamily{phv}\selectfont}

\usepackage[compact]{titlesec}

\usepackage{sectsty}
\allsectionsfont{\fontfamily{phv}\selectfont}

\usepackage{tocloft}
\newcommand{\cftXfont}{\fontfamily{phv}\selectfont}
\newcommand{\cftXleader}{\fontfamily{phv}\selectfont}
\newcommand{\cftXpagefont}{\fontfamily{phv}\selectfont}

\usepackage{mdwlist}

% \usepackage{fancyhdr}
% \fancypagestyle{plain}{%
% 	\clearpage
% 	\fancyfoot{}
% 	\fancyfoot[CE,CO]{\fontfamily{phv}\selectfont\thepage}
% 	\renewcommand{\headrulewidth}{0 pt}
% 	\renewcommand{\footrulewidth}{0 pt}
% }

\usepackage{opcit}

\begin{document}
\fontfamily{phv}\selectfont

\title{schades \\ School Administration e-System \\ \large{A back-end
approach to absence administration} \\ \large{Final Paper}} \author{Author:
Michal Sudwoj, 6bG, Kantonsschule Rychenberg \\ Supervision: Urs M\"{u}ller,
Kantonsschule Rychenberg}
\date{Submitted: December 1, 2009}
\maketitle
\thispagestyle{empty}

\newpage
\thispagestyle{empty}
\mbox{}

\newpage
\tableofcontents

\chapter{Introduction}

\section{How To Use This Paper}
Acronyms and specialist terms written in \textit{italic} are explained in
Appendix \ref{glossary} Glossary.

\section{Aims}
This project aims to create an \textit{absence unit} administration online
appliction, provide developer and user clarity, functionality and flexibility, so that it
can be integrated into the school \textit{intranet} with ease.

\section{History}
A way of managing the \textit{absence units}\footnote{Cf.
\ref{situation} Current Situation} exists, but it lacks clarity for students.
Some years ago a final paper\cite{gubler} attempting to create a clear absence
unit administration system was written, but sadly the application was not
introduced due to the lack of portability.

\section{Current Situation}
\label{situation}
The current system is a mixture of an online
\textit{MySQL}\footnote{Cf. \ref{mysql} MySQL} database, where the absence data
is stored, and a local \textit{FileMaker} database, where the data is managed and evaluated.
Information input is done over an \textit{intranet} page. The local database is
kept updated by manually running an import script. \textit{Absence unit}
administrators evaluate the data manually, printing and distributing large amounts forms or
letters in order to inform the pupils. Students can only see their current
amount of \textit{absence units} by retrieving a table held by the form teacher
or by requesting a printout from the administrator.

\chapter{Solution}

\section{Creation Process}
I wanted to create a useful application as my final paper. Most of my projects
are born from the desire to make things better. During consultations with Mr
M\"{u}ller, the subject of absence administration came up. This had been a
great issue in my form, and so I decided to take the project. Judging by the
amount of people asking me about the state of my final paper, I believe I made
the right choice.

During development, the programme was re-written three times as it got more
and more complex: first from one file into two, then into \textit{class} files
(modularisation), and lastly \textit{AJAX} was introduced.

\section{Technologies Used}

\subsection{MySQL}
\label{mysql}
\textit{MySQL} is ``the most popular \textit{Open Source} \textit{SQL} database
management system.''\cite{mysql} The school database is a \textit{MySQL} database.

\subsection{PHP}
\label{php}
\textit{PHP}, a retronym for \textit{PHP}: \textit{Hypertext}
\textit{Preprocessor}, ``is a widely-used \textit{open source} general-purpose
[\textit{server-side}] scripting language that is especially suited for web
development and can be embedded into \textit{HTML}.''\cite{php_what}
\textit{PHP} is commonly used to fetch data from a database and to display it,
or to manipulate data and return it in another format(e.g.\ a \textit{PDF}
document). The school \textit{intranet} as well as the timetable use this
technology, it was therefore wise to employ the same language.

\subsection{AJAX}
\label{ajax}
\textit{AJAX} ``(shorthand for asynchronous \textit{JavaScript} and
\textit{XML}) is a group of interrelated web development techniques used on the
\textit{client-side} to create interactive web applications.''\cite{ajax}
\textit{AJAX} allows for the loading of data without refreshing the whole web
page, thus drastically reducing loading time.

\subsection{PDF}
\label{pdf}
\textit{PDF} (Portable Document Format) is ``a system for storing and moving documents
between computers that only allows the contents to be viewed or printed, or a
document created using this system.''\cite{pdf} Being portable and wide-spread,
it makes a great tool for writing printable documents.

\subsection{\LaTeX{}}
\label{latex}
\textit{\LaTeX{}} ``is a high-quality typesetting system, \ldots the de facto standard
for the communication and publication of scientific documents.''\cite{latex}
This paper is written in \textit{\LaTeX{}}.

\section{Programmes Used}
\subsection{XAMPP}
\label{xampp}
\textit{XAMPP}\footnote{acronym for X (cross-platform), \textit{Apache},
\textit{MySQL}, \textit{PHP},
\textit{Perl}}\footnote{\url{http://apachefriends.com}} is an open source
software bundle consisting mainly of the \textit{Apache} web server, the
\textit{MySQL} database and \textit{PHP}.

\subsection{libTidy}
\label{tidy}
The \textit{libTidy}\footnote{\url{http://tidy.sourceforge.net/}} library and
the \textit{HTMLTidy} extension for \textit{PHP} help clean-up
\textit{PHP}-generated \textit{HTML} code, that is it closes unclosed
\textit{tags}, adds required \textit{elements} and indents code.

\subsection{TCPDF}
\label{tcpdf}
\textit{TCPDF}\footnote{\url{http://www.tcpdf.net}} is a \textit{PHP} class to
create \textit{PDF} files. It is written in pure \textit{PHP}, making it
system-independent, and it includes functions to turn \textit{HTML} code into
\textit{PDF} documents.

\subsection{Notepad++}
\label{npp}
\textit{Notepad++}\footnote{\url{http://notepad-plus.sourceforge.net/}} is a
more advanced version of the \textit{Notepad} application for
\textit{Windows\texttrademark}. It includes syntax-highlighting and
autocompletion. This was my main development environment on
\textit{Windows\texttrademark}.

\subsection{eclipse}
\label{eclipse}
\textit{eclipse}\footnote{\url{http://www.eclipse.org}} is a popular and
portable \textit{IDE}. Its many plug-ins allow for eased development in
different programming languages.

\subsection{SourceForge}
\label{sf}
\textit{SourceForge}\footnote{\url{http://www.sourceforge.net}} is a free
hosting solution for open source programmes. It is where the source code of this
application is published.

\section{Concepts and Features}

\subsection{Structure}
The files adding up to the application can be pidgeon-holed into the following
categories:
\begin{itemize}
  \item \textbf{Class:} represents an object and provides an interface for interaction
or groups together functions of a similar use
  \item \textbf{Script:} operates unseen for the end user, usually computing or writing output to display
  \item \textbf{Web Page:} provides the user interface
  \item \textbf{Setting:} configuration files
  \item \textbf{Documentation}
  \item \textbf{Resources} such as images or templates.
\end{itemize}
\begin{tabular}{ | l | l | l | }
\hline
\cellcolor[gray]{0.7}Classes & \cellcolor[gray]{0.7}Scripts & \cellcolor[gray]{0.7} Web Pages \\
\hline class/Absence.php & abs\_saver.php & index.php \\ \hline
class/Course.php & ajax\_excuse.php & login.php \\ \hline
class/Date.php & ajax\_main.php & \cellcolor{black}\\ \hline
class/DB.php & ajax\_paper.php & \cellcolor[gray]{0.7}Settings \\ \hline
class/Excuse.php & ajax\_queue.php & settings/settings.php \\ \hline
class/Form.php & ajax\_related.php & settings/styles.css \\ \hline
class/Paper.php & ajax\_schedule.php & settings/db/mat.php \\ \hline
class/Person.php & ajax\_student.php & \cellcolor{black}\\ \hline
class/Printer.php & ajax\_teacher.php & \cellcolor[gray]{0.7}Documentation \\ \hline
class/Queue.php & ajax.js & agpl.txt \\ \hline
\cellcolor{black} & excuse\_action.php & doc/document.tex \\ \hline
\cellcolor[gray]{0.7}Resources & logout.php & doc/ref.bib \\ \hline
img/agplv3-155x51.png & paper\_set.php & \cellcolor{black}\\ \hline
img/logo.jpg & queue\_action.php & \cellcolor[gray]{0.7}Libraries\\ \hline
\cellcolor{black} & \cellcolor{black} & /lib/tcpdf \\ \hline
\end{tabular}

\subsection{Absence Unit Calculating Algorithm}
One of the biggest challenges in the making of this application was the
development of a fair, accurate and reliable alogithm to automatically compute
the number of \textit{absence units} a student received. It is not enough to simply
count each single lesson the student was absent, since lengthy periods of uninterrupted absence cost
a maximum of four \textit{absence units}.

To solve this problem two concepts were introduced. Firstly, a student is
neither assumed to have attended each lesson nor not to have attended.
Secondly, the idea of blocks was introduced. A block is the longest period of
proved possible uninterrupted absence in a school week. Put simply, the
alogrithm searches each week for lessons where the student was absent and
assumes that the student was also absent in-between these lessons unless he or
she has been proven (i.e.\ a lesson has been marked as) present.

This has an obvious disadvantage for students, as it violates the
presumption of innocence (presence). This sacrifice is neccessary to guarantee a
fair evaluation even when not all absences are entered. An illustrative example:
a student falls ill and misses a day of school. One of his teachers does not
notice that he is not there and, believing him to be present, does not enter any absences since the system
assumes by default that all students are present. The system then gives the
student eight absence units, since he missed two blocks of five lessons each.
That would not happen in the current system, since the unentered absence would
still be counted to the block, making it a block of eleven lessons.

Of course, neither system is perfect. One could take the best of both and create
a system based on statistic evaluation and probability, but that would require
much more time than was available.

\subsection{Queue}
A new feature is the queue. It displays a list of actions requiring user
interaction, e.g.\ promotion to a higher level. The idea is based on \textit{FaceBook}
``Notifications'', which display latest activities concerning the user.

\section{Licencing}
I decided to release this programme under the \textit{AGPL} v3\footnote{The
full text of the licence is available from
\url{http://www.fsf.org/licensing/licenses/agpl-3.0.html}}, an
\textit{open source licence}, as it is my belief that \textit{open source}
software has advantages over \textit{closed source} software. Firstly, it
allows other developers to build upon one's work. Secondly, I believe
\textit{open source} software to be more secure, as more pressure is exerted on
the developer to patch any security holes.

\chapter{Conclusion}

\section{Future}
The application still lacks some features that were planned and it
needs to be integrated into the school \textit{intranet}. The programme,
although designed to minimize the effects of no being used, will not
work without the support of the school staff. I therefore ask all staff members
to consider trying out the system when it is released to the public.

I hope that the modular character
of the project will allow for faciliated continuation. I would imagine that a
personalised calendar, a form register, a homework planner, a school diary or
course administration software could be coded with ease; the creation of a
grade administration application would be more challenging. I hope to extend
this project in the above mentioned directions myself.

\section{Aknowledgements}
I wish to thank Mr Urs M\"{u}ller, my supervisor and tutor, for his aid and
co-operation, as well as Mr Hans-Peter Pleisch for explaining me the current
\textit{absence unit} administration system and the tasks of an administrator. I would
also like to thank my family for their supporting spirit when things did not go
as planned, and my teachers and my form for taking interest in this project.
\appendix

\chapter{Glossary}
\label{glossary}
\begin{basedescript}{\desclabelstyle{\fontfamily{phv}\selectfont}}

\item[Absence unit] A unit of measure for a students absence. A lesson of
absence usually corresponds to one absence unit.

\item[AGPL] Affero General Public License. An \textit{open source licence}
designed for web applications.

\item[AJAX] Cf. \ref{ajax} AJAX.

\item[Apache] Apache \textit{HTTP} Server. A very popular web server.

\item[Client-sided script] Scripts that are executed on the
client machine (e.g.\ PC) and have access to the client's resources. Cf.
\textit{server-side}.

\item[Class] A template or blueprint defining the properties and methods
(actions) that objects of the class have.

\item[Closed source software] Software that is ``neither free nor \textit{open
source}.''\cite{closed_source_wiki} Cf. \textit{open source}.

\item[eclipse] Cf. \ref{eclipse} eclipse.

\item[element] Cf. \textit{HTML element}.

\item[FaceBook] A popular social networking website
\footnote{\url{http://www.facebook.com/}}.

\item[FileMaker] A proprietary database system.

\item[HTML] Hypertext Markup Language. The language web pages
are written in.

\item[HTML element] Consists of a start tag, content, and an end tag. Cf.
\textit{HTML tag}.

\item[HTML tag] Content enclosed in tags is treated accordingly to the type of
tag used. Tags start with the \textless character and end in the \textgreater
character, e.g.\ \textless{}table\textgreater.

\item[HTMLTidy] Cf. \ref{tidy} libTidy.

\item[HTTP] \textit{Hypertext} Transfer Protocol. 

\item[Hypertext] Cf. \textit{HTML}.

\item[IDE] Integrated Development Environment. An application or
software bundle consisting of ``a source code editor, a compiler and/or an
interpreter, build automation tools [and] a debugger.''\cite{ide_wiki}

\item[Intranet] A computer network used to share information within an
organisation.

\item[JavaScript] A scripting language

\item[\LaTeX{}] Cf. \ref{latex} \LaTeX{}.

\item[libTidy] Cf. \ref{tidy} libTidy.

\item[MySQL] Cf. \ref{mysql} MySQL.

\item[Notepad] A native text editor for \textit{Windows\texttrademark}.

\item[Notepad++] Cf. \ref{npp} Notepad++.

\item[Open source software] Software that complies to the Open Source
Definition\footnote{\url{http://opensource.org/docs/osd}}. This software must
be free and its source code available to the public. Cf. \textit{closed source}.

\item[Open source licence] A licence which allows for free modification and
redistribution of a programmes source code. Cf. \textit{open source}.

\item[PDF] Portable Document Format. Cf. \ref{pdf} PDF.

\item[Perl] A general-purpose programming language.

\item[PHP] Cf. \ref{php} PHP.

\item[Preprocessor] A programme whose output is another programmes input.

\item[Server-sided script] Scripts that are executed on the server and have
access to server resources. Cf. \textit{client-side}.

\item[SourceForge] Cf. \ref{sf} SourceForge.

\item[SQL] Stuctured Query Language. A computer language used for the managing
of databases.

\item[Tag] Cf. \textit{HTML tag}.

\item[TCPDF] Cf. \ref{tcpdf} TCPDF.

\item[Windows\texttrademark] A popular operating system from Microsoft Corp.

\item[XAMMP] Cf. \ref{xampp} XAMPP.

\item[XML] eXtensible Markup Language. A computer language used for the
structuring of data.

\end{basedescript}

\chapter{Programmes Used in Development}
\section{Platform Independent}
\begin{itemize}

\item Texclipse Plugin 1.3.0, 25/11/2009, eclipse update site
\url{http://texlipse.sourceforge.net/}

\end{itemize}

\section{Windows}
\begin{itemize}

\item XAMPP for Windows, 03/05/2009,
\url{http://ch2.php.net/get/php-5.3.1.tar.bz2/from/a/mirror}

\item Notepad++ 5.3.1, 03/05/2009,
\url{http://sourceforge.net/projects/notepad-plus/files/notepad%2B%2B%20releases%20binary/npp%205.3.1%20bin/npp.5.3.1.Installer.exe/download}

\item MikTeX 2.7, 25/11/2009, \url{http://miktex.org/2.7/setup}

\end{itemize}

\section{Linux}
\begin{itemize}

\item XAMPP for Linux 1.7.2, 10/08/2009,
\url{http://sourceforge.net/projects/xampp/files/XAMPP%20Linux/xampp-linux-1.7.2.tar.gz/download}

\item XAMPP for Linux Development Package 1.7.2, 10/08/2009,
\url{http://sourceforge.net/projects/xampp/files/XAMPP%20Linux/xampp-linux-devel-1.7.2.tar.gz/download}

\item Eclipse for PHP Developers, Eclipse Galileo 3.5.1 (SR1),
\url{http://www.eclipse.org/downloads/download.php?file=/technology/epp/downloads/release/galileo/SR1/eclipse-php-galileo-SR1-linux-gtk.tar.gz&url=http://mirror.switch.ch/eclipse/technology/epp/downloads/release/galileo/SR1/eclipse-php-galileo-SR1-linux-gtk.tar.gz&mirror_id=63}

\item PHP 5.3.1 (inculding HTMLTidy), 25/11/2009,
\url{http://ch2.php.net/get/php-5.3.1.tar.bz2/from/a/mirror}

\end{itemize}

\renewcommand{\bibname}{\chapter{Sources}\section{Primary Sources}}
\bibliographystyle{plain}
\bibliography{ref}

\chapter{Source Code}
The most recent source code is availible for download from
\url{http://www.sourceforge.net/projects/schades}.

\lstlistoflistings
\newpage\section{class/Absence.php}
\lstinputlisting{../class/Absence.php}

\newpage\section{class/Calendar.php}
\lstinputlisting{../class/Calendar.php}

\newpage\section{class/Course.php}
\lstinputlisting{../class/Course.php}

\newpage\section{class/Date.php}
\lstinputlisting{../class/Date.php}

\newpage\section{class/DB.php}
\lstinputlisting{../class/DB.php}

\newpage\section{class/Excuse.php}
\lstinputlisting{../class/Excuse.php}

\newpage\section{class/Form.php}
\lstinputlisting{../class/Form.php}

\newpage\section{class/Paper.php}
\lstinputlisting{../class/Paper.php}

\newpage\section{class/Period.php}
\lstinputlisting{../class/Period.php}

\newpage\section{class/Person.php}
\lstinputlisting{../class/Person.php}

\newpage\section{class/Printer.php}
\lstinputlisting{../class/Printer.php}

\newpage\section{class/Queue.php}
\lstinputlisting{../class/Queue.php}

\newpage\section{abs\_saver.php}
\lstinputlisting{../abs_saver.php}

\newpage\section{ajax\_excuse.php}
\lstinputlisting{../ajax_excuse.php}

\newpage\section{ajax\_main.php}
\lstinputlisting{../ajax_main.php}

\newpage\section{ajax\_queue.php}
\lstinputlisting{../ajax_queue.php}

\newpage\section{ajax\_related.php}
\lstinputlisting{../ajax_related.php}

\newpage\section{ajax\_schedule.php}
\lstinputlisting{../ajax_schedule.php}

\newpage\section{ajax\_student.php}
\lstinputlisting{../ajax_student.php}

\newpage\section{ajax\_teacher.php}
\lstinputlisting{../ajax_teacher.php}

\newpage\section{ajax.js}
\lstinputlisting{../ajax.js}

\newpage\section{excuse\_action.php}
\lstinputlisting{../excuse_action.php}

\newpage\section{index.php}
\lstinputlisting{../index.php}

\newpage\section{login.php}
\lstinputlisting{../login.php}

\newpage\section{logout.php}
\lstinputlisting{../logout.php}

\newpage\section{paper\_set.php}
\lstinputlisting{../paper_set.php}

\newpage\section{settings/settings.php}
\lstinputlisting{../settings/settings.php}

\newpage\section{settings/styles.css}
\lstinputlisting{../settings/styles.css}

\newpage\section{settings/db/mat.php}
\lstinputlisting{../settings/db/mat.php}

\end{document}
